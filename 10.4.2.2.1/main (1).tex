\documentclass[12pt]{article}
\usepackage{amsmath}
\newcommand{\myvec}[1]{\ensuremath{\begin{pmatrix}#1\end{pmatrix}}}
\newcommand{\mydet}[1]{\ensuremath{\begin{vmatrix}#1\end{vmatrix}}}
\newcommand{\solution}{\noindent \textbf{Solution: }}
\providecommand{\brak}[1]{\ensuremath{\left(#1\right)}}
\providecommand{\norm}[1]{\left\lVert#1\right\rVert}
\let\vec\mathbf

\title{Qudratic Equation}
\author{Saipreet Pattjoshi (spattjoshi@sriprakashschools.com)}

\begin{document}
\maketitle
\section*{10$^{th}$ Maths - Chapter 4}
This is Problem-2 from Exercise 4.2
\begin{enumerate}
\item John and Jivanti together have 45 marbles. Both of them lost 5 marbles each, and the product of the number of marbles they now have is 124. We would like to find out how many marbles they have to start with. \\
\solution \\
Given Data:
\myvec{x^2-45x+324=0}\\
\\This can also be written as:
\myvec{x=\frac{-b\pm\sqrt{b^2-4ac}}{2a}}\\
\begin{align}
\myvec{x=\frac{-b\pm\sqrt{b^2-4ac}}{2a}}\\
\myvec{x=\frac{45\pm\sqrt{-45^2-4 \times 1\times324}}{2 \times 1}}\\
\myvec{x=\frac{45+\sqrt{2025-1296}}{2}}\\
\myvec{x=\frac{45+\sqrt{729}}{2}}\\
\end{align}
1st condition
\begin{align}
\myvec{x=\frac{45+27}{2}}\\
\myvec{x=\frac{72}{2}}\\
x=36
\end{align}
2nd Condition
\begin{align}
\myvec{x=\frac{45-27}{2}}\\
\myvec{x=\frac{18}{2}}\\
x=9
\end{align}


type your solutions using these commands
	

\end{enumerate}



\end{document}